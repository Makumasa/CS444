\documentclass[letterpaper,10pt,titlepage,draftclsnofoot,onecolumn]{article}

\usepackage{graphicx}                                        
\usepackage{amssymb}                                         
\usepackage{amsmath}                                         
\usepackage{amsthm}                                          
\usepackage{alltt}                                           
\usepackage{float}
\usepackage{color}
\usepackage{url}
\usepackage{hyperref}
\usepackage{listings}
\usepackage{longtable}
\usepackage{balance}
\usepackage[TABBOTCAP, tight]{subfigure}
\usepackage{enumitem}
\usepackage{pstricks, pst-node}
\usepackage[T1]{fontenc}
\usepackage{geometry}
\geometry{textheight=8.5in, textwidth=6in}


\newcommand{\cred}[1]{{\color{red}#1}}
\newcommand{\cblue}[1]{{\color{blue}#1}}

\newcommand{\toc}{\tableofcontents}

\def\name{Morgan Patch, Mark Bereza}
\def\subj{CS 444 Project 3: Encypted Block Device}

% The following metadata will show up in the PDF properties
\hypersetup{
    colorlinks = false,
    urlcolor = black,
    pdfauthor = {\name},
    pdftitle = {\subj},
    pdfsubject = {CS 444 Project 3},
    pdfpagemode = UseNone
}

\parindent = 0.0 in
\parskip = 0.1 in

\begin{document}

\begin{titlepage}
	\centering
	{\scshape\huge \subj \par}
	\vspace{1.5cm}
	{\LARGE\bfseries \name\par}

	\vfill
	
	{\large\bfseries Abstract\par}
	\vspace{0.5cm}
	
	For this project, we created a simple block device driver for RAM that maintains its data in an encrypted state.
	
	\vfill
	{\large \today\par}
\end{titlepage}

\tableofcontents
\newpage

\section{Design}

Although the details of implementation were messy and involved a lot of debugging, the overall design of the block device driver was fairly simple. We would use the example code for a simple block driver from the "LWM.net Porting Drivers to 2.6 Series" and update it to work on the 3.19.2 version of the Linux Kernel. From there, we would utilize the Linux crypto API to perform EBC encryption on all data to be written and perform a symmetric decryption on all data to be read. The one caveat would be for data being read from the disk that has not yet been written to, which should be read as-is without being decrpyted. Finally, the key for the encryption would be used as a module parameter so that it may be defined by the user when the module is loaded.

\section{Questions}

\subsection{What do you think the main point of this assignment is?}

The point of this assignment seems to be three-fold:
\begin{enumerate}
\item The first point is for us to familiarize ourselves with and gain experience implementing block device drivers for the Linux kernel.

\item The second point is for us to familiarize ourselves with and gain experience utilizing the Linux crypto API to perform encryption and decryption. Related to this goal is the one involving learning how cryptography may be utilized to secure data on an arbitrary digital storage medium.

\item The third point is for us to familiarize ourselves with and gain experience creating modules for the Linux kernel that can be attached at runtime.
\end{enumerate}

\subsection{How did you personally approach the problem?}

We began by first trying to get a working implementation of the simple block driver without cryptography. The one provided in the aforementioned source was created for version 2.6 of the kernel and several of the function calls, header files, and request struct members were deprecated. Specifically, here is a list of the deprecated references and their updated equivalents:\\
\begin{tabular}{rcl}
request\_t & $\rightarrow$ & struct request \\
elv\_next\_request() & $\rightarrow$ & blk\_fetch\_request \\
req->sector & $\rightarrow$ & blk\_rq\_pos() \\
req->current\_nr\_sectors & $\rightarrow$ & blk\_rq\_cur\_sectors() \\
req->buffer & $\rightarrow$ & bio\_data(request->bio) \\
end\_request() & $\rightarrow$ & \_\_blk\_end\_request\_cur() \\
\end{tabular}\\ \\
Beyond these changes, several additional header files had to be included due to certain references being moved since the 2.6 version of the kernel. After all this was done, the block driver finally compiled successfully. 

After verifying that it worked correctly, the next step was to implement cryptography. Research was done on the Linux crypto API and several pieces were added onto the existing simple block driver to implement encryption/decryption. First, encrypt/decrpyt keys were defined and made to be a module parameter. Second, members were added to the sbd\_device struct in order to hold the crypto\_ciphers and an array to keep track of what blocks have been written to and what blocks have not. The latter member was created to help prevent attempts to decrypt uninitialized data. Next, code was added to the sbd\_init() function to allocate the ciphers and associate the appropriate keys with them. Any keys not specified by the user through the module would be set to a default value. Finally, the sbd\_transfer() function was changed to encrypt the contents of the write buffer block by block before writing the data to the device. Areas of the device that were written to were flagged as such so the driver knew that reads from those locations should first be decrypted. As for reading, a check was first done to see if the requested memory location was initialized or not. Uninitialized data would be copied into the read buffer as-is, whereas encrypted data would first be decrypted block by block.

Beyond that, several printk() statements were added in order to verify that the driver was functioning correctly and to aid in debugging. These included printing when functions started and ended, the type of operation being done (read vs. write), the location and number of bytes for each operation, and a hex dump of the data before/after cryptography for each operation.

\subsection{How did you ensure your solution was correct?}

In order to prove that the device worked correctly, we tested:
\begin{enumerate}
\item That a file system could be created and mounted on the device
\item That the data was being read correctly
\item That it was stored in a different form on the disk than in the requests
\item That the individual blocks were encrypted correctly. 
\end{enumerate}
\par To prove the data was being read correctly (i.e. it worked as a block device), we first formatted the drive using mkfs.ext4 /dev/sbd0, then mounted the drive with mount /dev/sbd0 /mnt, then wrote several files onto the drive, and read them back out to confirm they were unchanged; we then unmounted the drive, remounted it, and checked that the data was still there. In order to prove the data was being transformed, we added a function to our code which would, each time sbd\_transfer() was called to read or write data, print out both the plaintext and encrypted forms of that block of data as a series of hexidecimal pairs; the fact that the plaintext and encrypted data were different proved the existence of a transformation. Finally, to prove the encryption was correct, we copied several of these blocks in both their plaintext and encrypted form, from both reads and writes; by performing the encryption/decryption ourselves using available online implementations of AES, we could confirm that they provided the same data as our code did, proving that our implementation of AES was correct.

Additionally, to ensure that the encryption was sensitive to the key being used, we loaded the module using a decrypt key different from the encrypt key and verified that the data being read was garbage through the hexdumps. In fact, this prevented the device from being able to properly mount a filesystem.

\subsection{What did you learn?}
By doing this assignment, we learned several things:
\begin{enumerate}
\item We learned how to update deprecated kernel code to work on newer versions. In particular, some of the deprecated symbols used in the 2.6 simple block driver could only be replaced with their modern equivalents by determining which version dropped said deprecated symbol and looking at commits to the relevant file for that version. This approach allowed us to determine how to replace the req->buffer code.
\item We learned how to create a Linux module and plug it in during runtime.
\item We learned the differences between different transformation algorithms provided by the Linux crypto API. In particular, we first attempted to use CBC, but found that it was extremely difficult to correctly implement for operations of varying sizes and also required the use of scatterlists, which themsleves are complicated constructs. In light of this, we instead opted to use EBC encryption.
\item We learned how to use scatterlists whilst flirting with the idea of using CBC-AES as our encryption/decryption algorithm.
\end{enumerate}

\subsection{How should the TA evaluate your work?}
In order to evaluate our work, the TA should take the following steps:
\begin{enumerate}
\item Examine our source code, provided below, to see if our approach to encryption makes sense.
\item Clone the linux yocto repository:\\
\texttt{git clone git://git.yoctoproject.org/linux-yocto-3.19}
\item Switch to the v3.19.2 tag:\\
\texttt{cd linux-yocto-3.19/} \\
\texttt{git checkout tags/v3.19.2}
\item Apply the patch file provided with this submission:\\
\texttt{git apply assignment3.patch}
\item Copy the configuration file from scratch:\\
\texttt{cp /scratch/files/config-3.19.2-yocto-standard ./.config}
\item Copy the core image file from scratch:\\
\texttt{cp /scratch/files/core-image-lsb-sdk-qemux86.ext4 .}
\item Setup environment variables:\\
\texttt{source /scratch/files/environment-setup-i586-poky-linux} (if using bash/zsh) \\
OR \\
\texttt{source /scratch/files/environment-setup-i586-poky-linux.csh} (if using tcsh/csh)
\item Edit the configuration to use our scheduler:\\
\texttt{make menuconfig} \\
Then go to Device drivers$\rightarrow$Block devices. From there, highlight "Simple Block Device" and hit 'M' to tell the kernel to build it as a module. Save this configuration to .config and exit menuconfig.
\item Build the kernel:\\
\texttt{make all -j4}
\item Run the kernel using QEMU:\\
\texttt{qemu-system-i386 -nographic -kernel arch/x86/boot/bzImage -drive file=core-image}\\
\texttt{-lsb-sdk-qemux86.ext4,if=virtio -enable-kvm -net user,hostfwd=tcp::5540-:22 -net}\\
\texttt{ nic -usb -localtime --no-reboot --append "root=/dev/vda rw console=ttyS0 debug"}
\item Verify that the kernel boots and you can log in to root.
\item Open a new terminal at the root kernel directory and copy the sbd.ko kernel module onto the emulated machine:\\
\texttt{scp -P 5540 drivers/blocksbd.ko root@localhost:sbd.ko}
\item In QEMU, load the driver module with some encrypt key. By not specifying a decrypt key, the decrypt key will default to be the same as the encrypt key:\\
\texttt{insmod ./sbd.ko encrypt\_key=121,53,161,29,3,65,9,196,167,167,221,203,208,40,190,}\\
\texttt{186}
\item Create a filesystem on the sbd0 device that should now be available:\\
\texttt{mkfs.ext4 /dev/sbd0}
\item Mount the filesystem:\\
\texttt{mount /dev/sbd0 /mnt}
\item Create a file:\\
\texttt{cd /mnt}\\
\texttt{touch sample}
\item Write some bytes to the file:\\
\texttt{dd if=/dev/urandom of=sample count=8 bs=4096}
\item Print some of the contents and save them somewhere not on the device:\\
\texttt{head sample}
\item Unmount the filesystem:\\
\texttt{cd /}\\
\texttt{umount /mnt}
\item Now we remount the filesystem to verify that the data remained the same:\\
\texttt{mount /dev/sbd0 /mnt}\\
\texttt{cd /mnt}\\
\texttt{head sample}\\
Compare this output with the output you saved earlier.
\item Now we need to verify that the encryption is working correctly. To do this, first find a write operation in the dmesg output (should start with "sbd: Operation type: write") and copy one or more lines from the before-encryption hexdump (should start with "sbd: Data before encryption:"). Additionally, find the same line(s) in the after-encryption hexdump (should start with "sbd: Data after encryption") and copy them, as well. Then, enter the unenceypted lines (along with the key provided to the module earlier) into a tool capable of AES ECB encryption/decryption, such as \url{http://aes.online-domain-tools.com/}. Verify that the output produced by this tool matches the encrypted output from the hexdump.
\item Now we do essentially the same thing, but this time for a read operation. Make sure that the before-decryption hexdunp for the read operation selected isn't a bunch of zeros; this indicates that the data is not encrypted. Verify using the same AES ECB tool that decryption is being performed correctly.
\item Finally, we must verify that using the incorrect key will not work. To to do this, we must unload the module and reload it, specifiying a decrypt key different from the encrypt key:\\
\texttt{cd /}\\
\texttt{umount /mnt}\\
\texttt{rmmod sbd}\\
\texttt{insmod ./sbd.ko encrypt\_key=121,53,161,29,3,65,9,196,167,167,221,203,208,40,190,}\\
\texttt{186 decrypt\_key=71,144,180,225,211,54,25,234,222,104,210,145,47,167,159,107}\\ \\
\texttt{mkfs.ext4 /dev/sbd0}\\
\texttt{mount /dev/sbd0 /mnt}\\
The mount operation should fail, giving an error message along the lines of:\\
\texttt{mount: wrong fs type, bad option, bad superblock on /dev/sbd0, missing codepage or helper program, or other error}\\
This is expected, since using the wrong decrypt key causes the kernel to incorrectly interpret the filesystem's metadata.
\end{enumerate}

\section{Version Control Log}

\begin{tabular}{lp{12cm}}
  \label{tabular:legend:git-log}
  \textbf{acronym} & \textbf{meaning} \\
  V & \texttt{version} \\
  tag & \texttt{git tag} \\
  MF & Number of \texttt{modified files}. \\
  AL & Number of \texttt{added lines}. \\
  DL & Number of \texttt{deleted lines}. \\
\end{tabular}

\bigskip

\iflanguage{ngerman}{\shorthandoff{"}}{}
\begin{longtable}{|rlllrrr|}
\hline \multicolumn{1}{|c}{\textbf{V}} & \multicolumn{1}{c}{\textbf{tag}}
& \multicolumn{1}{c}{\textbf{date}}
& \multicolumn{1}{c}{\textbf{commit message}} & \multicolumn{1}{c}{\textbf{MF}}
& \multicolumn{1}{c}{\textbf{AL}} & \multicolumn{1}{c|}{\textbf{DL}} \\ \hline
\endhead

\hline \multicolumn{7}{|r|}{} \\ \hline
\endfoot

\hline% \hline
\endlastfoot

\hline 1 &  & 2017-10-19 & Add Linux Yocto at tag 3.19.2. & 48441 & 19142068 & 0 \\
\hline 2 &  & 2017-11-07 & Added sbd.c using code from LVM.net & 1 & 189 & 0 \\
\hline 3 &  & 2017-11-07 & made changes to drivers/block Makefile & 1 & 1 & 0 \\
\hline 4 &  & 2017-11-07 & drivers/block/Kconfig now references BLK\_DEV\_SBD & 1 & 3 & 0 \\
\hline 5 &  & 2017-11-07 & Fixed some bugs in sbd.c & 1 & 2 & 7 \\
\hline 6 &  & 2017-11-07 & Changes to sbd.c & 1 & 1 & 0 \\
\hline 7 &  & 2017-11-09 & Add include line to SBD. & 1 & 3 & 2 \\
\hline 8 &  & 2017-11-12 & got sbd.c to compile & 1 & 115 & 123 \\
\hline 9 &  & 2017-11-12 & Add crypto setup and free. & 3 & 42 & 8 \\
\hline 10 &  & 2017-11-12 & Add writing encrypted data. & 1 & 27 & 4 \\
\hline 11 &  & 2017-11-12 & Add reading decrypted data. & 1 & 16 & 2 \\
\hline 12 &  & 2017-11-12 & Correct sizes. & 1 & 2 & 2 \\
\hline 13 &  & 2017-11-12 & Added kernel messages & 1 & 19 & 0 \\
\hline 14 &  & 2017-11-12 & Added kernel messages & 1 & 0 & 1 \\
\hline 15 &  & 2017-11-12 & Fixed compiler warnings & 1 & 18 & 3 \\
\hline 16 &  & 2017-11-12 & Minor change to sbd.c & 1 & 4 & 4 \\
\hline 17 &  & 2017-11-12 & Move crypto above registration in init. & 1 & 31 & 31 \\
\hline 18 &  & 2017-11-12 & Use correct algorithm. & 1 & 1 & 1 \\
\hline 19 &  & 2017-11-12 & Add 16-byte default key. & 1 & 3 & 2 \\
\hline 20 &  & 2017-11-12 & minor changes & 1 & 36 & 27 \\
\hline 21 &  & 2017-11-12 & Re-add key and make arrays on heap. & 1 & 16 & 10 \\
\hline 22 &  & 2017-11-12 & Move malloc and free outside if. & 1 & 7 & 18 \\
\hline 23 &  & 2017-11-12 & fixed compiler warnings/errors & 1 & 4 & 4 \\
\hline 24 &  & 2017-11-12 & Now prints data before/after encryption/decryption & 1 & 36 & 4 \\
\hline 25 &  & 2017-11-14 & Fix some merge issues. & 1 & 6 & 6 \\
\hline 26 &  & 2017-11-14 & Skip uninitialized memory. & 1 & 18 & 0 \\
\hline 27 &  & 2017-11-14 & now does not decrypt uninitialized blocks & 1 & 19 & 10 \\
\hline 28 &  & 2017-11-14 & Rename uninit function to match. & 1 & 2 & 2 \\
\hline 29 &  & 2017-11-15 & code cleanup and IV is not freed & 1 & 58 & 63 \\
\hline 30 &  & 2017-11-15 & new approach to checking for uninitialized blocks & 1 & 63 & 57 \\
\hline 31 &  & 2017-11-15 & Move to EBC. & 1 & 29 & 43 \\
\hline 32 &  & 2017-11-15 & Move error checking into place. & 1 & 2 & 2 \\
\hline 33 &  & 2017-11-15 & Remove unnecessary module parameters. & 1 & 0 & 3 \\
\hline 34 &  & 2017-11-15 & Added writeup files & 2 & 626 & 0 \\
\hline 35 &  & 2017-11-15 & Remove unnecessary temp buffers. & 1 & 9 & 24 \\
\hline 36 &  & 2017-11-15 & Use two keys if given. & 1 & 68 & 24 \\
\hline 37 &  & 2017-11-15 & more work done to writeup & 1 & 112 & 192 \\
\hline 38 &  & 2017-11-15 & fleshed out more sections of writeup & 1 & 10 & 56 \\
\hline 39 &  & 2017-11-15 & Fix key lengths. & 1 & 2 & 2 \\
\end{longtable}

\section{Work Log}

\begin{itemize}
    \item Tuesday, 2017-11-07
    
        \begin{itemize}
            \item Looked at assignment description.
			\item Copied sbd.c code from LVM.net.
			\item Edited kconfig and make files to recognize the driver.
			\item Began work on editing the sbd code to compile for 3.19.2.
        \end{itemize}
    
    \item Thursday, 2017-11-09
    
        \begin{itemize}
            \item Continued to work on getting sbd.c to compile
            \item Researched Linux crypto API.
        \end{itemize}
        
    \item Sunday, 2017-11-12
    
        \begin{itemize}
            \item Got sbd.c to compile.
			\item Added crypto setup to sbd\_init()
			\item Added decryption to reads.
			\item Added encryption to writes.
			\item Added crypto key.
			\item Added a bunch of printk() messages.
        \end{itemize}
        
    \item Tuesday, 2017-11-14
    
        \begin{itemize}
            \item Determined why CBC was only correctly decrypting first request.
			\item Added logic to prevent decryption of uninitialized blocks.
			\item Started writeup.
        \end{itemize}
		
	\item Wednesday, 2017-11-15
		\begin{itemize}
			\item Switched to EBC encryption.
			\item Fixed logic for determining uninitialized blocks.
			\item Removed unneeded intermediate buffers.
			\item Now can handle two keys: one for encrypting and the other for decrypting.
			\item Loaded driver as module successfully.
			\item Successfully mounted filesystem on device and wrote to/read from files.
			\item Completed write up.
			\item Created patch file.
			\item Turned in assignment.
		\end{itemize}
\end{itemize}

\section*{Appendix 1: Source Code}
\subsection{sbd.c}

\lstset{language=C}

\begin{lstlisting}
/*
* A sample, extra-simple block driver.
*
* Copyright 2003 Eklektix, Inc.  Redistributable under the terms
* of the GNU GPL.
*/

#include <linux/module.h>
#include <linux/moduleparam.h>
#include <linux/init.h>
#include <linux/blkdev.h>
#include <linux/elevator.h>
#include <linux/bio.h>

#include <linux/kernel.h> /* printk() */
#include <linux/fs.h>     /* everything... */
#include <linux/errno.h>  /* error codes */
#include <linux/types.h>
#include <linux/vmalloc.h>
#include <linux/genhd.h>
#include <linux/blkdev.h>
#include <linux/hdreg.h>

#include <linux/crypto.h>
#include <linux/err.h>
#include <linux/scatterlist.h>
#include <linux/random.h>

#include <linux/mm.h>
#include <linux/sysinfo.h>

#define SBD_ENCRYPTED 0
#define SBD_PLAINTEXT 1

MODULE_LICENSE("Dual BSD/GPL");
static char *Version = "1.3";

static int major_num = 0;
static int logical_block_size = 512;
static int nsectors = 1024;  /* How big the drive is */
static u8 default_key[] = {0x26, 0xe7, 0x0a, 0x0c, 0xbc, 0xce, 0x4a, 0x5c, 0x0b,
        0x0b, 0xd1, 0xd8, 0xf7, 0xce, 0x2d, 0xf7};
static const unsigned int default_keylen = 16;
static u8 encrypt_key[] = {};
static unsigned int ekeylen = 0;
module_param_array(encrypt_key, byte, &ekeylen, 0);
static u8 decrypt_key[] = {};
static unsigned int dkeylen = 0;
module_param_array(decrypt_key, byte, &dkeylen, 0);


/*
* We can tweak our hardware sector size, but the kernel talks to us
* in terms of small sectors, always.
*/
#define KERNEL_SECTOR_SIZE 512

/*
* Our request queue.
*/
static struct request_queue *Queue;

/*
* The internal representation of our device.
*/
static struct sbd_device {
        unsigned long size;
        spinlock_t lock;
        u8 *data;
        u8 *written;
        struct gendisk *gd;
        struct crypto_cipher *encrypt;
        struct crypto_cipher *decrypt;
} Device;

static void sbd_print_buffer(u8 *buffer,
                             unsigned long len,
                             unsigned long offset,
                             int type)
{
        int i;
        for (i = 0; i < len; ++i) {
                if ((i % 16) == 0) {
                        if (type == SBD_PLAINTEXT)
                                printk("\nPLAINTEXT, BYTES %06lu-%06lu: ",
                                       i + offset, i + offset + 15);
                        else
                                printk("\nENCRYPTED, BYTES %06lu-%06lu: ",
                                       i + offset, i + offset + 15);
                }
                printk("%02x ", (unsigned int) *(buffer + i));
        }
        printk("\n");
}

/*
* Handle an I/O request.
*/
static void sbd_transfer(struct sbd_device *dev, unsigned long sector,
                         unsigned long nsect, char *buffer, int write)
{
        int i;
        unsigned long offset = sector * logical_block_size;
        unsigned long nbytes = nsect  * logical_block_size;
        // Encrypt and decrypt have the same block size
        unsigned int crypt_blksize = crypto_cipher_blocksize(dev->encrypt);
        u8 *memloc = dev->data + offset;

        printk(KERN_NOTICE "sbd: Starting sbd_transfer()...\n");
        printk(KERN_NOTICE "sbd: offset        = %lu\n", offset);
        printk(KERN_NOTICE "sbd: nbytes        = %lu\n", nbytes);
        printk(KERN_NOTICE "sbd: crypt_blksize = %u\n", crypt_blksize);

        if ((offset % crypt_blksize) != 0) {
                printk(KERN_ERR
                       "sbd: Offset not aligned to crypto block size\n");
                return;
        } else if (nbytes % crypt_blksize != 0) {
                printk(KERN_ERR
                       "sbd: Nbytes not aligned to crypto block size\n");
                return;
        } else if ((offset + nbytes) > dev->size) {
                printk(KERN_ERR "sbd: Beyond-end write (%ld %ld)\n",
                                    offset, nbytes);
                return;
        }

        if (write) {
                int i;

                printk(KERN_NOTICE "sbd: Operation type: write\n");

                for (i = 0; i < nbytes; i += crypt_blksize)
                        crypto_cipher_encrypt_one(dev->encrypt, memloc + i,
                                                  buffer + i);

                for (i = (offset / crypt_blksize);
                     i < ((offset + nbytes) / crypt_blksize);
                     ++i)
                        (dev->written)[i] = 1;

                printk(KERN_NOTICE "sbd: Data before encryption:");
                sbd_print_buffer(buffer, nbytes, offset, SBD_PLAINTEXT);
                printk(KERN_NOTICE "sbd: Data after encryption:");
                sbd_print_buffer(memloc, nbytes, offset, SBD_ENCRYPTED);
        } else {
                printk(KERN_NOTICE "sbd: Operation type: read\n");

                for (i = 0; i < nbytes; i += crypt_blksize) {
                        if ((dev->written)[(offset + i) / crypt_blksize]) {
                                crypto_cipher_decrypt_one(dev->decrypt,
                                                          buffer + i,
                                                          memloc + i);
                        } else {
                                memcpy(buffer + i, memloc + i,
                                       crypt_blksize);
                        }
                }
                printk(KERN_NOTICE "sbd: Data before decryption:");
                sbd_print_buffer(memloc, nbytes, offset, SBD_ENCRYPTED);
                printk(KERN_NOTICE "sbd: Data after decryption:");
                sbd_print_buffer(buffer, nbytes, offset, SBD_PLAINTEXT);
        }

        printk(KERN_NOTICE "sbd: Ending sbd_transfer()...\n");
}

static void sbd_request(struct request_queue *q)
{
        struct request *req;
        printk(KERN_NOTICE "sbd: Starting sbd_request()...\n");
        req = blk_fetch_request(q);
        while (req != NULL) {
                if (req == NULL || (req->cmd_type != REQ_TYPE_FS)) {
                        printk(KERN_NOTICE "Skip non-CMD request\n");
                        __blk_end_request_all(req, -EIO);
                        continue;
                }
                sbd_transfer(&Device, blk_rq_pos(req), blk_rq_cur_sectors(req),
                             bio_data(req->bio), rq_data_dir(req));
                if (!__blk_end_request_cur(req, 0)) {
                        req = blk_fetch_request(q);
                }
        }
        printk(KERN_NOTICE "sbd: Ending sbd_request()...\n");
}

/*
 * The HDIO_GETGEO ioctl is handled in blkdev_ioctl(), which
 * calls this. We need to implement getgeo, since we can't
 * use tools such as fdisk to partition the drive otherwise.
 */
int sbd_getgeo(struct block_device * block_device, struct hd_geometry * geo)
{
        long size;

        printk(KERN_NOTICE "sbd: Starting sbd_getgeo()...\n");

        /* We have no real geometry, of course, so make something up. */
        size = Device.size * (logical_block_size / KERNEL_SECTOR_SIZE);
        geo->cylinders = (size & ~0x3f) >> 6;
        geo->heads = 4;
        geo->sectors = 16;
        geo->start = 0;

        printk(KERN_NOTICE "sbd: Ending sbd_getgeo()...\n");

        return 0;
}

/*
* The device operations structure.
*/
static struct block_device_operations sbd_ops = {
        .owner = THIS_MODULE,
        .getgeo = sbd_getgeo
};

static int __init sbd_init(void)
{
        struct crypto_cipher *encrypt, *decrypt;
        unsigned int encrypt_blksize, decrypt_blksize;
        unsigned int ret;

        printk(KERN_NOTICE "sbd: Starting sbd_init()...\n");

        /*
         * First, init our crypto.
         */

        // If both keys are given, use them; if one is given, it is used for
        // both; if no keys are given, use the default for both

        if (ekeylen != 0 && dkeylen != 0 && ekeylen != dkeylen) {
                printk(KERN_WARNING "sbd: keys must be the same length\n");
                goto out;
        }

        encrypt = crypto_alloc_cipher("aes", 0, 0);
        if (IS_ERR(encrypt)) {
                printk(KERN_WARNING "sbd: unable to allocate cipher\n");
                goto out;
        }
        encrypt_blksize = crypto_cipher_blocksize(encrypt);

        if (ekeylen == 0 && dkeylen == 0) {
                ret = crypto_cipher_setkey(encrypt, default_key, default_keylen);
        } else if (ekeylen == 0) {
                ret = crypto_cipher_setkey(encrypt, decrypt_key, dkeylen);
        } else {
                ret = crypto_cipher_setkey(encrypt, encrypt_key, ekeylen);
        }
        if (ret) {
                printk(KERN_NOTICE "sbd: crypto_blkcipher_setkey() failed\n");
                goto out;
        }

        Device.encrypt = encrypt;

        decrypt = crypto_alloc_cipher("aes", 0, 0);
        if (IS_ERR(decrypt)) {
                printk(KERN_WARNING "sbd: unable to allocate cipher\n");
                goto out;
        }
        decrypt_blksize = crypto_cipher_blocksize(decrypt);

        if (ekeylen == 0 && dkeylen == 0) {
                ret = crypto_cipher_setkey(decrypt, default_key, default_keylen);
        } else if (dkeylen == 0) {
                ret = crypto_cipher_setkey(decrypt, encrypt_key, ekeylen);
        } else {
                ret = crypto_cipher_setkey(decrypt, decrypt_key, dkeylen);
        }
        if (ret) {
                printk(KERN_NOTICE "sbd: crypto_blkcipher_setkey() failed\n");
                goto out;
        }

        Device.decrypt = decrypt;

        if (encrypt_blksize != decrypt_blksize) {
                printk(KERN_WARNING "sbd: block lengths must be the same\n");
                goto out;
        }

        /*
        * Set up our internal device.
        */
        Device.size = nsectors * logical_block_size;
        spin_lock_init(&Device.lock);
        Device.data    = vzalloc(Device.size);
        Device.written = vzalloc(Device.size / encrypt_blksize);
        if (Device.data == NULL)
                return -ENOMEM;
        /*
        * Get a request queue.
        */
        Queue = blk_init_queue(sbd_request, &Device.lock);
        if (Queue == NULL)
                goto out;
        blk_queue_logical_block_size(Queue, logical_block_size);
        /*
        * Get registered.
        */
        major_num = register_blkdev(major_num, "sbd");
        if (major_num <= 0) {
                printk(KERN_WARNING "sbd: unable to get major number\n");
                goto out;
        }
        /*
        * And the gendisk structure.
        */
        Device.gd = alloc_disk(16);
        if (!Device.gd)
                goto out_unregister;
        Device.gd->major        = major_num;
        Device.gd->first_minor  = 0;
        Device.gd->fops         = &sbd_ops;
        Device.gd->private_data = &Device;
        strcpy(Device.gd->disk_name, "sbd0");
        set_capacity(Device.gd, nsectors);
        Device.gd->queue        = Queue;
        add_disk(Device.gd);

        printk(KERN_NOTICE "sbd: Ending sbd_init()...\n");

        return 0;

out_unregister:
        unregister_blkdev(major_num, "sbd");
out:
        vfree(Device.written);
        vfree(Device.data);
        crypto_free_cipher(Device.encrypt);
        crypto_free_cipher(Device.decrypt);
        printk(KERN_NOTICE "sbd: Ending sbd_init()...\n");

        return -ENOMEM;
}

static void __exit sbd_exit(void)
{
        printk(KERN_NOTICE "sbd: Starting sbd_exit()...\n");

        del_gendisk(Device.gd);
        put_disk(Device.gd);
        unregister_blkdev(major_num, "sbd");
        blk_cleanup_queue(Queue);
        vfree(Device.written);
        vfree(Device.data);
        crypto_free_cipher(Device.encrypt);
        crypto_free_cipher(Device.decrypt);

        printk(KERN_NOTICE "sbd: Ending sbd_exit()...\n");
}

module_init(sbd_init);
module_exit(sbd_exit);

\end{lstlisting}

\end{document}
